\documentclass[10pt]{article}
\usepackage{fullpage}
\usepackage{hyperref}
\usepackage{amsmath}
\usepackage{amsthm}
%\usepackage{mathtools}
%\usepackage{amsfonts}
%\usepackage{amsthm}
%\usepackage{setspace}
%\usepackage{algorithm}
%\usepackage{algpseudocode}
%\usepackage{times}
%\usepackage{etoolbox}% http://ctan.org/pkg/etoolbox
\usepackage{enumerate}
\usepackage{bbm}
\usepackage{mathtools}

\usepackage[utf8]{inputenc}
\usepackage[english]{babel}
\usepackage{minted}

\newtheorem{myprop}{Proposition}

\newcommand{\A}{\mathcal{A}}
\newcommand{\X}{\mathcal{X}}
\newcommand{\G}{\mathcal{G}}
\newcommand{\floor}[1]{\lfloor #1 \rfloor}
\newcommand{\ceil}[1]{\lceil #1 \rceil}
\newcommand{\ind}{\mathbbm{1}}
\newcommand{\abs}[1]{\left| #1 \right|}
\newcommand{\Unif}{\mathrm{Unif}}
\newcommand\defeq{\stackrel{\mathclap{\normalfont\tiny\mbox{def}}}{=}}
\newcommand\rgets{\stackrel{\mathclap{\normalfont\tiny\mbox{\$}}}{\gets}}

\title{Security analysis of \texttt{passgen}}
\date{}
\author{Stephen Tu \\ steve33671@gmail.com}

\begin{document}
\maketitle

\section{Introduction}
\texttt{passgen} is a simple tool for generating cryptographically random
passwords for various online services including banking services such as
American Express, Fidelity, etc. While conceptually the problem of generating a
random password is quite trivial (e.g. draw random characters from the allowed
character set), there are a few subtle implementation issues that arise when
ensuring that the resulting implementation is indeed cryptographically secure.

The purpose of this document is to outline \texttt{passgen}'s approach to these
issues. The discussion here is neither groundbreaking nor technically
difficult; this is meant as a easy, fun read. \textit{Disclaimer}: the author is
not a security expert by any means, so comments/corrects are very much welcome
and appreciated.

\section{Secure random string generation}
A string/password is nothing more than a relabelling (bijection) between $\A$,
the set of characters allowed (the alphabet), and the integer set $\{0, ...,
\abs{\A}-1\}$. Therefore, it might be tempting to generate passwords with the 
simple snippet below:
\begin{minted}{python}
from random import randint
from string import letters, digits

def generate(length):
    charset = letters + digits # [a-zA-Z0-9]
    return ''.join(charset[randint(0, len(charset)-1)] for _ in xrange(length))
\end{minted}
This, unfortunately, is not a secure implementation. In fact, the python
doc\footnote{\url{https://docs.python.org/2/library/random.html}} for the
\verb|random| module explicitly states:
\begin{verbatim}
Warning: The pseudo-random generators of this module should not be used for
security purposes.  Use os.urandom() or SystemRandom if you require a
cryptographically secure pseudo-random number generator.
\end{verbatim}
This should not be surprising; the reason is that for efficiency (and
portability) purposes, the default random number generator in standard
libraries is typically a \emph{pseudo random number generator} (PRNG).  A PRNG
is completely deterministic \emph{once} the underlying state is known.  One of
the most well known PRNGs is the \emph{linear congruential generator}, which
given a state $s_n$, generates the $n+1$-th random number via the simple formula:
\begin{equation*}
  s_{n+1} \gets a s_n + c \mod{m}
\end{equation*}
with well known choices for $a, c,
m$.\footnote{\url{http://en.wikipedia.org/wiki/Linear_congruential_generator}}
Thus, while PRNGs are fast, if an attacker can guess the state, then she can
predict with 100\% accuracy the resulting coin flips from the PRNG.

\subsection{OS support for entropy generation} To deal with this problem when
security matters, operating systems provide first class support for generating
true randomness; the python doc warning hints at this. On Linux, two devices
\texttt{/dev/random} and \texttt{/dev/urandom} are available. The man
page\footnote{\url{http://man7.org/linux/man-pages/man4/random.4.html}}
provides a detailed explanation of the difference. For the purposes of this
discussion, we won't care which one is used, since both expose the same
interface to our tool. The interface exposed is very unix-y: both
\texttt{/dev/random} and \texttt{/dev/urandom} look like any other read-only
file to our application, and so we can use the standard POSIX \verb|open| and
\verb|read| filesystem API to access the entropy.

In some sense we are already done; the hard question of generating true,
cryptographically secure randomness has been punted to the kernel and so we
simply need to take it on faith and use one of the random devices. However, two
issues remain:
\begin{enumerate}
  \item Since \verb|/dev/{,u}random| looks like a file, the smallest unit of data
    we can read out of it is a byte (8 bits). How can we use this primitive to induce
    a uniform distribution on any arbitrary discrete set $\X$?
  \item Many sites (sadly) have higher level requirements on passwords such as 
    a password must contain at least one letter and digit or cannot contain
    more than 5 consecutive digits, etc. Programatically, we can model each
    requirement as a predicate $p$ which takes a password and returns true iff
    the input password satisfies the requirements. How do we efficiently sample
    a password \emph{uniformly} at random without resorting to enumeration?
\end{enumerate}
We will focus the remainder of this document on answering these two questions.
Our main tool will be the simple, elegant idea of \emph{rejection sampling}. 

\section{Sampling from arbitrary discrete uniform distributions}
Let's set up some notation. Let $\X$ denote a finite set over some arbitrary
discrete alphabet, and let $Y$ be a random variable. We write $Y \sim
\Unif(\X)$ to indicate that $Y$ has the uniform distribution on $\X$.  That is,
the probability mass function of $Y$ is $\Pr(Y=y) = \frac{1}{\abs{\X}}
\ind_{\X}(y)$.

From our discussion above, we will assume that \verb|/dev/{,u}random| provides
us with a secure implementation of $Y_k \sim \Unif(\{0, ..., 2^{8k} - 1\}) \defeq \G_k$ for some $k \geq 1$.
%
The goal will be to produce an implementation of $Z \sim \Unif(\X)$ for any
$\X$ given only $Y_k \sim \G_k$. Wlog we assume $\X = \{0, ..., n-1\}$ for some $n
\geq 1$.

At first, it may be tempting to simply pick the smallest $k$ such that $2^{8k}
\geq n$, take $Z = Y_k \mod{n}$ and call it a day. It is
important to emphasize: \textbf{this is absolutely incorrect!}.
%
To see why, take $n=3$ and $k=1$. Now consider $N_i = \{ 0 \leq x < 256 : x = i \mod{3}
\}$ for $i \in \{0,1,2\}$. It is not hard to see that $\abs{N_0} = 86$ but
$\abs{N_1}=\abs{N_2} = 85$. Therefore, $\Pr(Z=0) = 0.3359$ but
$\Pr(Z=1)=\Pr(Z=2)=0.3320$!

However, this strategy \emph{does} work if $n$ is a power of two, that is
$n=2^l$ for some $l \geq 0$. The easiest way to see this is to consider the
binary representation of $Y_k$-- $Y_k$ is nothing more than a random bitstring
from $\{0,1\}^{8k}$. Recalling that taking a number mod $n$ when $n$ is a power
of two is equivalent to taking a bitwise \verb|and| with $n-1$, then since
$n=2^{l}$, $Z$ is simply the first $l$ bits of $Y_k$. Since $Y_k$ is uniform,
the marginal distribution of the first $l$ bits (the distribution of $Z$) is
also uniform.

We're almost there. We've turned our sampler from $G_k$ into a sampler from
$Z_l = \Unif(\{0, ..., 2^l - 1\}) \defeq H_l$ for any $l \geq 0$. The remaining
step to create an arbitrary sampler from $\Unif(\{0,...,n-1\})$ is to apply a
rejection sampling step. Consider the following algorithm:
\begin{enumerate}
  \item Set $l$ such that $2^l \geq n$.
  \item Sample $Y \rgets Z_l$.
  \item If $Y < n$ then \verb|accept| $Y$, otherwise \verb|reject| $Y$ and go back to step 2.
\end{enumerate}
\begin{myprop}
The rejection sampler above satisfies $\Pr(Y=i | \normalfont{\texttt{accept}})
= \frac{1}{n}$ if $0 \leq i < n$, and zero otherwise.
\end{myprop}
\begin{proof}
By the definition of conditional probability, we have
\begin{align*}
  \Pr(Y=i|\texttt{accept}) = \frac{\Pr(Y=i, \texttt{accept})}{\Pr(\texttt{accept})} 
  = \frac{\Pr(Y=i)\Pr(\texttt{accept}|Y=i)}{\Pr(\texttt{accept})}
\end{align*}
Noting that $\Pr(\texttt{accept}) = \Pr(Y<n) = \frac{n}{2^l}$, $\Pr(Y=i)=\frac{1}{2^l}$, and 
$\Pr(\texttt{accept}|Y=i) = \ind_{\{0, ..., n-1\}}(i)$, we have 
that $\Pr(Y=i|\texttt{accept}) = \frac{1/2^l}{n/2^l} = \frac{1}{n} \ind_{\{0, ..., n-1\}}(i)$
which yields the claim.
\end{proof}
\end{document}
