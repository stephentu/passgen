\documentclass[10pt]{article}
\usepackage{fullpage}
\usepackage{hyperref}
\usepackage{amsmath}
\usepackage{amsthm}
%\usepackage{mathtools}
%\usepackage{amsfonts}
%\usepackage{amsthm}
%\usepackage{setspace}
%\usepackage{algorithm}
%\usepackage{algpseudocode}
%\usepackage{times}
%\usepackage{etoolbox}% http://ctan.org/pkg/etoolbox
\usepackage{enumerate}
\usepackage{bbm}
\usepackage{mathtools}

\usepackage[utf8]{inputenc}
\usepackage[english]{babel}
\usepackage{minted}

\newtheorem{myprop}{Proposition}

\newcommand{\A}{\mathcal{A}}
\newcommand{\X}{\mathcal{X}}
\newcommand{\G}{\mathcal{G}}
\newcommand{\HH}{\mathcal{H}}

\newcommand{\floor}[1]{\lfloor #1 \rfloor}
\newcommand{\ceil}[1]{\lceil #1 \rceil}
\newcommand{\ind}{\mathbbm{1}}
\newcommand{\abs}[1]{\left| #1 \right|}
\newcommand{\Unif}{\mathrm{Unif}}
\newcommand{\supp}{\mathrm{supp}}
\renewcommand{\Pr}{\mathrm{Pr}}
\newcommand\defeq{\stackrel{\mathclap{\normalfont\tiny\mbox{def}}}{=}}
\newcommand\rgets{\stackrel{\mathclap{\normalfont\tiny\mbox{\$}}}{\gets}}

\title{Security analysis of \texttt{passgen}}
\date{}
\author{Stephen Tu \\ \texttt{tu.stephenl@gmail.com}}

\begin{document}
\maketitle

\section{Introduction}
\texttt{passgen}\footnote{\url{https://github.com/stephentu/passgen}} is a
simple tool for generating cryptographically random passwords for various
online services including banking services such as American Express, Fidelity,
etc. While conceptually the problem of generating a random password is quite
trivial (e.g. draw random characters from the allowed character set), there are
a few subtle implementation issues that arise when ensuring that the resulting
implementation is indeed cryptographically secure.

The purpose of this document is to outline \texttt{passgen}'s approach to these
issues. The discussion here is neither groundbreaking nor technically
difficult; this is meant as a easy, fun read. \textit{Disclaimer}: the author is
not a security expert by any means, so comments/corrects are very much welcome
and appreciated.

\section{Secure random string generation}
A string/password is nothing more than a relabelling (bijection) between $\A$,
the set of characters allowed (the alphabet), and the integer set $\{0, ...,
\abs{\A}-1\}$. Therefore, it might be tempting to generate passwords with the 
simple snippet below:
\begin{minted}{python}
from random import randint
from string import letters, digits

def generate(length):
    charset = letters + digits # [a-zA-Z0-9]
    return ''.join(charset[randint(0, len(charset)-1)] for _ in xrange(length))
\end{minted}
This, unfortunately, is not a secure implementation. In fact, the python
doc\footnote{\url{https://docs.python.org/2/library/random.html}} for the
\verb|random| module explicitly states:
\begin{verbatim}
Warning: The pseudo-random generators of this module should not be used for
security purposes.  Use os.urandom() or SystemRandom if you require a
cryptographically secure pseudo-random number generator.
\end{verbatim}
This should not be surprising; the reason is that for efficiency (and
portability) purposes, the default random number generator in standard
libraries is typically a \emph{pseudo random number generator} (PRNG).  A PRNG
is completely deterministic once the underlying state is known.  One of
the most well known PRNGs is the \emph{linear congruential generator}, which
given a state $s_n$, generates the $n+1$-th random number via the simple formula:
\begin{equation*}
  s_{n+1} \gets a s_n + c \mod{m}
\end{equation*}
with well known choices for $a, c,
m$.\footnote{\url{http://en.wikipedia.org/wiki/Linear_congruential_generator}}
Thus, while PRNGs are fast, if an attacker can guess the state $s_n$, then she
can predict with 100\% accuracy the resulting coin flips (e.g. $s_{n+k}$ for $k
\geq 1$) from the PRNG.

\subsection{OS support for entropy generation} To deal with this problem when
security matters, operating systems provide first class support for generating
randomness; the python doc warning hints at this. On Linux, two devices
\texttt{/dev/random} and \texttt{/dev/urandom} are available. The man
page\footnote{\url{http://man7.org/linux/man-pages/man4/random.4.html}}
provides a detailed explanation of the difference. For the purposes of this
discussion, we won't care which one is used, since both expose the same
interface to our tool. The interface exposed is very unix-y: both
\texttt{/dev/random} and \texttt{/dev/urandom} look like any other read-only
file to our application, and so we can use the standard POSIX \verb|open| and
\verb|read| filesystem API to access the entropy.

\paragraph{Issues.} In some sense we are already done; the hard question of generating true,
cryptographically secure randomness has been punted to the kernel and so we
simply need to take it on faith and use one of the random devices. However, several
issues remain:
\begin{enumerate}
  \item Since \verb|/dev/{,u}random| looks like a file, the smallest unit of data
    we can read out of it is a byte (8 bits). How can we use this primitive to induce
    a uniform distribution on any arbitrary discrete set $\X$?
  \item Given a uniform distribution on an arbitrary discrete set $\X$, how can
    we generate a uniform sampler on $\X^{m,n} \defeq \X^m \cup \X^{m+1} \cup ... \cup \X^{n}$ 
    where $m \leq n$?
  \item Many sites (sadly) have higher level requirements on passwords such as 
    a password must contain at least one letter and digit or cannot contain
    more than 5 consecutive digits, etc. We can model each
    requirement as a predicate $p$ which takes a password and returns true iff
    the input password satisfies the requirements. How do we efficiently sample
    a password \emph{uniformly} at random without resorting to enumeration?
\end{enumerate}
We will focus the remainder of this document on answering these questions.
Our main tool will be the simple, elegant idea of \emph{rejection sampling}. 

\section{Sampling from arbitrary discrete uniform distributions}
\label{sec:unif}
Let's set up some notation. Let $\X$ denote a finite set over some arbitrary
discrete alphabet, and let $Y$ be a random variable. We write $Y \sim
\Unif(\X)$ to indicate that $Y$ has the uniform distribution on $\X$.  That is,
the probability mass function of $Y$ is $\Pr(Y=y) = \frac{1}{\abs{\X}}$ for $y \in \X$.

From our discussion above, we will assume that \verb|/dev/{,u}random| provides
us with a secure implementation of $Y_k \sim \Unif(\{0, ..., 2^{8k} - 1\}) \defeq \G_k$ for some $k \geq 1$.
%
The goal will be to produce an implementation of $Z \sim \Unif(\X)$ for any
$\X$ given only $Y_k \sim \G_k$. Wlog we assume $\X = \{0, ..., n-1\}$ for some $n
\geq 1$.

At first, it may be tempting to simply pick the smallest $k$ such that $2^{8k}
\geq n$, take $Z = Y_k \mod{n}$ and call it a day. It is
important to emphasize: \textbf{this is absolutely incorrect!}.
%
To see why, take $n=3$ and $k=1$. Now consider $N_i = \{ 0 \leq x < 256 : x = i \mod{3}
\}$ for $i \in \{0,1,2\}$. It is not hard to see that $\abs{N_0} = 86$ but
$\abs{N_1}=\abs{N_2} = 85$. Therefore, $\Pr(Z=0) = 0.3359$ but
$\Pr(Z=1)=\Pr(Z=2)=0.3320$!

However, this strategy \emph{does} work if $n$ is a power of two, that is
$n=2^l$ for some $l \geq 0$. The easiest way to see this is to consider the
binary representation of $Y_k$-- $Y_k$ is nothing more than a random bitstring
from $\{0,1\}^{8k}$. Recalling that taking a number mod $n$ when $n$ is a power
of two is equivalent to taking a bitwise \verb|and| with $n-1$, then since
$n=2^{l}$, $Z$ is simply the first $l$ bits of $Y_k$. Since $Y_k$ is uniform,
the marginal distribution of the first $l$ bits (the distribution of $Z$) is
also uniform.

\paragraph{Algorithm.}
We're almost there. We've turned our sampler from $\G_k$ into a sampler from
$\HH_l \defeq \Unif(\{0, ..., 2^l - 1\})$ for any $l \geq 0$. The remaining step
to create an arbitrary sampler from $\Pi_{n} \defeq \Unif(\{0,...,n-1\})$ is to
apply a rejection sampling step.  Consider the following algorithm:
\begin{enumerate}
  \item Set $l$ such that $2^l \geq n$.
  \item Sample $Y \rgets \HH_l$.
  \item If $Y < n$ then \verb|accept| $Y$, otherwise \verb|reject| $Y$ and go back to step 2.
\end{enumerate}
\begin{myprop}
\label{prop:samp1}
The rejection sampler above satisfies $\Pr(Y=i | \normalfont{\texttt{accept}})
= \frac{1}{n}$ if $0 \leq i < n$, and zero otherwise.
\end{myprop}
\begin{proof}
By the definition of conditional probability, we have
\begin{align}
  \Pr(Y=i|\texttt{accept}) = \frac{\Pr(Y=i, \texttt{accept})}{\Pr(\texttt{accept})} 
  = \frac{\Pr(Y=i)\Pr(\texttt{accept}|Y=i)}{\Pr(\texttt{accept})} \label{eq:condprob}
\end{align}
Noting that $\Pr(\texttt{accept}) = \Pr(Y<n) = \frac{n}{2^l}$, $\Pr(Y=i)=\frac{1}{2^l}$, and 
$\Pr(\texttt{accept}|Y=i) = \ind_{\{0, ..., n-1\}}(i)$, we have 
that $\Pr(Y=i|\texttt{accept}) = \frac{1/2^l}{n/2^l} = \frac{1}{n} \ind_{\{0, ..., n-1\}}(i)$
which yields the claim.
\end{proof}
Note that in our proof, we did not use any property of the distribution $\HH_l$
other than the fact that the support of $\HH_l$ contains the support of $\Pi_n$.
In fact, we could have replaced $\HH_l$ with $\G_k$ for any $k$ such that $2^{8k}
\geq n$ and the proof remains valid. 
%This observation leads to two main observations, the first of which we will discuss now. 
Why did we use (prefer) the proposal distribution
$\HH_l$ over $\G_k$ when both are correct? The answer is $\HH_l$ is more \emph{efficient}
than $\G_k$, where efficient means the probability of accepting a sample from 
$\HH_l$ is greater than accepting a sample from $\G_k$. This is easy to see,
since 
\begin{equation*}
    \Pr_{Y \sim \HH_l}(\texttt{reject}) = \frac{1}{\abs{\supp(\HH_l) \setminus \supp(\Pi_n)}}
        \leq \frac{1}{\abs{\supp(\G_k)\setminus\supp(\Pi_n)}} = \Pr_{Y \sim \G_k}(\texttt{reject})
\end{equation*}

\section{Sampling from unions of variable length sets}
\label{sec:var}
From Section~\ref{sec:unif}, given a set $\X$ and an integer $k \geq 1$, it is
clear how to sample uniformly from $\X^k$: simply repeat the sampling algorithm
$k$ times. But what happens if we want to sample uniformly from $\X^{m,n}$,
which happens when we want to generate random passwords between say, 10 to 20
characters long? 

\paragraph{Algorithm.} Turns out, the simple algorithm below suffices. The idea is,
we first sample $D$, the length of the password to draw. Then, we use the value of
$D$ to draw a uniform sample from $\X^D$.
\begin{enumerate}
  \item $D \rgets k$, with $\Pr(D=k) = \frac{\abs{\X}^k}{\sum_{i=m}^{n} \abs{\X}^i}$ for $k \in \{m, ..., n\}$.
  \item $Y \rgets \Unif(\X^{D})$.
  \item Return $Y$ as random draw.
\end{enumerate}

\begin{myprop}
The algorithm above satisfies $\Pr(Y = \omega) = \frac{1}{ \sum_{i=m}^{n} \abs{X}^i }$, for $\omega \in \X^{m,n}$.
\end{myprop}
\begin{proof}
A straightforward computation yields
\begin{align*}
  \Pr(Y=\omega) = \sum_{k=m}^{n} \Pr(D=k, Y=\omega) = \Pr(D=\abs{\omega})\Pr(Y=\omega | D=\abs{\omega}) = \frac{\abs{X}^{\abs{\omega}}}{ \sum_{i=m}^{n} \abs{\X}^i } \cdot \frac{1}{ \abs{\X}^{\abs{\omega}} } = \frac{1}{ \sum_{i=m}^{n} \abs{X}^i }
\end{align*}
\end{proof}

\section{Sampling from predicate defined domains}
From Section~\ref{sec:var}, we can sample efficiently from $\Unif(\X^{m,n})$. Now, we assume
we are given a predicate $p : \X^{m,n} \longrightarrow \{ \texttt{T}, \texttt{F} \}$.
Define the set $\mathcal{V}$ as
\begin{align*}
  \mathcal{V} \defeq \{ \omega \in \X^{m,n} : p(\omega) = \texttt{T} \}
\end{align*}
Our goal is to sample from $\Unif(\mathcal{V})$. It turns out, we have already
seen a special case of this in Section~\ref{sec:unif}, where we used rejection
sampling. We can re-apply that technique here in almost the same way. 

\paragraph{Algorithm.} Consider the following sampler.
\begin{enumerate}
  \item $Y \rgets \Unif(\X^{m,n})$.
  \item If $p(Y) = \texttt{T}$, then \verb|accept| $Y$, otherwise \verb|reject| $Y$ and go back to step 2.
\end{enumerate}
\begin{myprop}
  The rejection sampler above satisfies $\Pr(Y=\omega|\normalfont{\texttt{accept}}) = \frac{1}{\abs{\mathcal{V}}}$ for $\omega \in \mathcal{V}$.
\end{myprop}
\begin{proof}
The proof is almost identical to Proposition~\ref{prop:samp1}. Here, the quantities of interest are
\begin{align*}
  \Pr(Y=\omega) = \frac{1}{\abs{\X^{m,n}}}, \qquad \Pr(\texttt{accept}) = \frac{\abs{\mathcal{V}}}{\abs{\X^{m,n}}}, \qquad \Pr(\texttt{accept}|Y=\omega) = \ind_{\omega \in \mathcal{V}}
\end{align*}
Plugging these quantities into (\ref{eq:condprob}) yields the result.
\end{proof}

\end{document}
